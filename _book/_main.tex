% Options for packages loaded elsewhere
\PassOptionsToPackage{unicode}{hyperref}
\PassOptionsToPackage{hyphens}{url}
\documentclass[
]{book}
\usepackage{xcolor}
\usepackage{amsmath,amssymb}
\setcounter{secnumdepth}{5}
\usepackage{iftex}
\ifPDFTeX
  \usepackage[T1]{fontenc}
  \usepackage[utf8]{inputenc}
  \usepackage{textcomp} % provide euro and other symbols
\else % if luatex or xetex
  \usepackage{unicode-math} % this also loads fontspec
  \defaultfontfeatures{Scale=MatchLowercase}
  \defaultfontfeatures[\rmfamily]{Ligatures=TeX,Scale=1}
\fi
\usepackage{lmodern}
\ifPDFTeX\else
  % xetex/luatex font selection
\fi
% Use upquote if available, for straight quotes in verbatim environments
\IfFileExists{upquote.sty}{\usepackage{upquote}}{}
\IfFileExists{microtype.sty}{% use microtype if available
  \usepackage[]{microtype}
  \UseMicrotypeSet[protrusion]{basicmath} % disable protrusion for tt fonts
}{}
\makeatletter
\@ifundefined{KOMAClassName}{% if non-KOMA class
  \IfFileExists{parskip.sty}{%
    \usepackage{parskip}
  }{% else
    \setlength{\parindent}{0pt}
    \setlength{\parskip}{6pt plus 2pt minus 1pt}}
}{% if KOMA class
  \KOMAoptions{parskip=half}}
\makeatother
\usepackage{longtable,booktabs,array}
\usepackage{calc} % for calculating minipage widths
% Correct order of tables after \paragraph or \subparagraph
\usepackage{etoolbox}
\makeatletter
\patchcmd\longtable{\par}{\if@noskipsec\mbox{}\fi\par}{}{}
\makeatother
% Allow footnotes in longtable head/foot
\IfFileExists{footnotehyper.sty}{\usepackage{footnotehyper}}{\usepackage{footnote}}
\makesavenoteenv{longtable}
\usepackage{graphicx}
\makeatletter
\newsavebox\pandoc@box
\newcommand*\pandocbounded[1]{% scales image to fit in text height/width
  \sbox\pandoc@box{#1}%
  \Gscale@div\@tempa{\textheight}{\dimexpr\ht\pandoc@box+\dp\pandoc@box\relax}%
  \Gscale@div\@tempb{\linewidth}{\wd\pandoc@box}%
  \ifdim\@tempb\p@<\@tempa\p@\let\@tempa\@tempb\fi% select the smaller of both
  \ifdim\@tempa\p@<\p@\scalebox{\@tempa}{\usebox\pandoc@box}%
  \else\usebox{\pandoc@box}%
  \fi%
}
% Set default figure placement to htbp
\def\fps@figure{htbp}
\makeatother
\setlength{\emergencystretch}{3em} % prevent overfull lines
\providecommand{\tightlist}{%
  \setlength{\itemsep}{0pt}\setlength{\parskip}{0pt}}
\usepackage[]{natbib}
\bibliographystyle{plainnat}
\usepackage{booktabs}
\usepackage{bookmark}
\IfFileExists{xurl.sty}{\usepackage{xurl}}{} % add URL line breaks if available
\urlstyle{same}
\hypersetup{
  pdftitle={PSY 607 Bayesian Analysis},
  pdfauthor={Sara Weston},
  hidelinks,
  pdfcreator={LaTeX via pandoc}}

\title{PSY 607 Bayesian Analysis}
\author{Sara Weston}
\date{Spring 2025}

\begin{document}
\maketitle

{
\setcounter{tocdepth}{1}
\tableofcontents
}
\chapter{Course description}\label{course-description}

\section{Structure}\label{structure}

\section{Materials}\label{materials}

\section{Schedule}\label{schedule}

\section{Grades}\label{grades}

\section{Policies}\label{policies}

\subsection{Abscences}\label{abscences}

\subsection{Communication}\label{communication}

If you have questions about course policies, have trouble submitting an assignment, or want to schedule a meeting, please email. I will make an effort to respond to emails within one business day. Note that I neither plan nor commit to checking email outside of normal business hours (9am-5pm, Mon-Fri).

If you are having trouble understanding a concept covered in class, please come to office hours, schedule a meeting with me, or ask for clarification during class periods. I will not explain course concepts over email.

Occasionally, I will send out announcements to the entire class via Canvas announcements. These will typically appear when you open Canvas, but you can update your Canvas settings to receive these announcements as emails. It is strongly recommended that you do so.

\subsection{Classroom}\label{classroom}

All members of the class (students and instructor) can expect to:

\emph{Participate and Contribute:} All students are expected to participate by sharing ideas and contributing to the learning environment. This entails preparing, following instructions, and engaging respectfully and thoughtfully with others.

While all students should participate, participation is not just talking, and a range of participation activities support learning. Participation might look like speaking aloud in the full class and in small groups and collaborating on homework assignments.

\emph{Expect and Respect Diversity:} All classes at the University of Oregon welcome and respect diverse experiences, perspectives, and approaches. What is not welcome are behaviors or contributions that undermine, demean, or marginalize others based on race, ethnicity, gender, sex, age, sexual orientation, religion, ability, or socioeconomic status. We will value differences and communicate disagreements with respect.

\emph{Help Everyone Learn:} Part of how we learn together is by learning from one another. To do this effectively, we need to be patient with each other, identify ways we can assist others, and be open-minded to receiving help and feedback from others. Don't hesitate to contact me to ask for assistance or offer suggestions that might help us learn better.

\subsection{Workload}\label{workload}

This is a 3-credit hour course, so you should expect to complete 120 hours of work for the course---an average of about 12 hours each week (this includes time in-class).

\subsection{Generative AI}\label{generative-ai}

\subsection{Plagiarism}\label{plagiarism}

\subsection{Accessibility}\label{accessibility}

\subsection{Basic needs}\label{basic-needs}

\subsection{Reporting obligations}\label{reporting-obligations}

\subsection{Campus emergencies}\label{campus-emergencies}

\chapter{Week 1: Introduction to Bayesian Analysis}\label{week-1-introduction-to-bayesian-analysis}

\section{Lecture 1: Probability}\label{lecture-1-probability}

\subsection{Slides}\label{slides}

\bibliography{book.bib,packages.bib}

\end{document}
